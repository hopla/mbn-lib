\chapter{Using the Library}
\section{Initialization}
Each MambaNet node the application provides has a special handle. All library functions that act from the node need this handle as argument, and all callback functions will receive it to determine from which MambaNet node the callback came from.

A handle can be obtained by calling mbnInit(), this function expects various information about the node as argument and will then initialize the node and request a MambaNet address. The following example illustrates how to initialize a node on Ethernet without custom objects:
\begin{verbatim}
  #include <mbn.h>

  struct mbn_interface *itf;
  struct mbn_handler *mbn;
  char error[MBN_ERRSIZE];

  // Information about the node
  struct mbn_node_info node = {
    0, 0,                            // MambaNetAddr, Services
    "MambaNet Library Example Node", // Description
    "Example #1",                    // Name
    0xFFFF, 0x0001, 0x0001           // UniqueMediaAccessID
    // other fields can be left blank
  };

  // Open the ethernet interface
  // The interface name is hardcoded in this example, but see
  // the section about interface modules for a better solution.
  if((itf = mbnEthernetOpen("eth0", error)) == NULL) {
    printf("Error opening ethernet device: %s", error);
    return;
  }

  // Initialize the MambaNet node
  if((mbn = mbnInit(&node, NULL, itf, error)) == NULL) {
    printf("Error initializing MambaNet node: %s", error);
    return;
  }
\end{verbatim}

After successful execution, \textit{mbn} points to a valid MambaNet node handle and can be used to attach callback functions and call library functions. In an actual application you would usually want to provide more information in \textit{node} and you would want to make the ethernet device configurable, this will be discussed later in this document.

Keep in mind that right after initialization, the MambaNet node does not have a validated address. You have to wait for the OnlineStatus() callback before sending (object) messages to the network.

\verb|mbnFree(mbn)| can be called to deallocate and remove the node from the network after initialization. This will also automatically deallocate the memory returned by \textit{mbnEthernetOpen()}, so the application doesn't need to keep track of the \textit{itf} pointer.




\section{Custom objects}
Custom node objects can be added to a MambaNet node at intialization by setting the \textit{NumberOfObjects} field in the \verb|mbn_node_info| structure passed to mbnInit(), and by providing a pointer to an array of \verb|mbn_object| structures to the \textit{objects} argument. A helper function is available to create this array without having to initialize all fields manually. The following example illustrates how to use this function and add the objects to the MambaNet node. Refer to the API documentation for \verb|mbn_object| and MBN\_OBJ() for more information.
\begin{verbatim}
  // Tell mbn.h to add MBN_OBJ() to our code
  #define MBN_VARARG
  #include "mhn.h"
  
  // allocate an array for 3 objects
  struct mbn_object objects[3];
  
  // object #1: sensor of type unsigned int
  objects[0] = MBN_OBJ(
    "Object #1", // name
    // sensor type,  freq, size (bytes), min, max, current value
    MBN_DATATYPE_UINT,  0,    2        ,   0, 512,           256,
    // actuator (none)
    MBN_DATATYPE_NODATA
  );
  
  // object #2: actuator of type float
  objects[1] = MBN_OBJ(
    "Object #2", // name
    // sensor (none)
    MBN_DATATYPE_NODATA,
    // actuator type, size,   min,  max, default, current
    MBN_DATATYPE_FLOAT,  4, -30.0, 10.0,     0.0,     0.0
  );
  
  // object #3: signed int sensor and float actuator
  objects[2] = MBN_OBJ(
    "Object #3",
    MBN_DATATYPE_SINT, 0, 4, -100, 100, 0,
    MBN_DATATYPE_FLOAT, 4, -100.0, 100.0, 0.0
  );
  
  // specify how many objects we have
  info.NumberOfObjects = 3;
  
  // initialize the MambaNet node
  mbn = mbnInit(&info, objects, itf, error);
\end{verbatim}

The library will then automatically reply to object messages from other nodes on the network. Interaction with these objects can be done through several functions and callbacks. The following example shows how you can link the actuator ``input'' of Object \#3 to the sensor ``output'' of the same object.
\begin{verbatim}
  int SetActuatorData(struct mbn_handler *mbn,
                      unsigned short object,
                      union mbn_data in) {
    union mbn_data out;
    
    // execute the default action if this is not object #3
    // (object counts from 0)
    // normally, an application should handle all defined objects
    if(object != 2)
      return 0;

    // set the sensor data to the actuator data
    // (converting float to signed int)
    out.SInt = (int)in.Float;
    mbnUpdateSensorData(mbn, 2, out);
    
    // always return 0, unless something went wrong
    return 0;
  }
  
  int main() {
    struct mbn_handler *mbn;
    // [initialization omitted, we know how that works already]
    
    // set the callback
    // (no, the double 'Set' here is not a typo)
    mbnSetSetActuatorDataCallback(mbn, SetActuatorData);
    
    // [do some work here to keep the program alive]
  }
\end{verbatim}



\section{Address table}
The library keeps track of all MambaNet nodes on the network and notifies the application whenever a change occurs through the AddressTableChange callback. It is also possible to browse through the address table at any time in the application using the mbnNextNode() function. The following example illustrates how this function can be used to output a list of online nodes:
\begin{verbatim}
  struct mbn_address_node *node;

  for(node=NULL; (node=mbnNextNode(mbn, node))!=NULL; ) {
    printf("%08lX -> %04X:%04X:%04X\n",
      node->MambaNetAddr, node->ManufacturerID,
      node->ProductID, node->UniqueIDPerProduct);
  }
\end{verbatim}

MambaNet nodes that are currently not on the network can not be detected, and will thus not be available in the list. Refer to the API documentation for \verb|mbn_address_node| for more information about which information is known.

The address table can also be used when receiving a message from another node on the network, and you want to know more about that node without having to send extra messages over the network. To do this, call the mbnNodeStatus() function. The following example shows how to ignore all incoming MambaNet messages from ManufacurerID 0xFFFF.
\begin{verbatim}
  int ReceiveMessage(struct mbn_handler *mbn,
                     struct mbn_message *msg) {
    struct mbn_address_node *node;
    
    node = mbnNodeStatus(mbn, msg->AddressFrom);
    if(node && node->ManufacturerID == 0xFFFF)
      return 1;
    else
      return 0;
  }
  
  // somewhere after initializing mbn...
  mbnSetReceiveMessageCallback(mbn, ReceiveMessage);
\end{verbatim}



\section{Interface modules}
The MambaNet protocol can be used over several different transport layers, the library handles this by providing ``interface modules'': modules that define and implement how the communication with the transport layer is done.

The current library has two built-in interface modules: Ethernet and TCP. As seen in the section discussing initialization, an interface module has to be initialized and opened before calling mbnInit(). Each module provides its own set of functions to do this.

The ethernet module provides a function to get the list of possible ethernet devices to use. The following example illustrates how to open the last available device on the machine. Refer to the documentation of the \verb|mbn_if_ethernet| struct and the mbnEthernetOpen() function for more information.
\begin{verbatim}
  // check for the availability of the ethernet module,
  // this is done at compile-time.
#ifndef MBN_IF_ETHERNET
# error No ethernet module available!
#else
  struct mbn_if_ethernet *devices, *last;
  struct mbn_interface *itf;
  char error[MBN_ERRSIZE];
  
  // get the list of devices
  if((devices = mbnEthernetIFList(error)) == NULL) {
    printf("Couldn't get device list: %s\n", error);
    return;
  }
  
  // browse to the last device in the list
  for(last=devices; devices->next!=NULL; last=devices->next)
    ;
  
  // open the interface
  if((itf = mbnEthernetOpen(last->name, error)) == NULL) {
    printf("Couldn't open device "%s": %s\n", last->name, error);
    return;
  }
  
  // free the device list
  mbnEthernetIFFree(devices);
#endif
\end{verbatim}

The TCP module can be initialized using the mbnTCPOpen() function, refer to the API documentation for an explanation on how to use it.

